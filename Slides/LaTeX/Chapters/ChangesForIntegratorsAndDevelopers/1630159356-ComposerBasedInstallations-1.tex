% ------------------------------------------------------------------------------
% TYPO3 Version 11.4 - What's New (English Version)
%
% @author	Michael Schams <schams.net>
% @license	Creative Commons BY-NC-SA 3.0
% @link		https://typo3.org/help/documentation/whats-new/
% @language	English
% ------------------------------------------------------------------------------
% Feature | 94996 | Consider all Composer installed extensions as active
% Deprecation | 94996 | In Composer Mode all Extensions should be installed with Composer

\begin{frame}[fragile]
	\frametitle{Changes for Integrators and Developers}
	\framesubtitle{Composer-based Installations}

	% decrease font size for code listing
	\lstset{basicstyle=\fontsize{7}{9}\ttfamily}

	\begin{itemize}
		\item The file "\texttt{PackageStates.php}" has become obsoluete in
			Composer-based installations
		\item All extensions added to the system by Composer are now considered
			to be active
		\item Therefore, the Extension Manager and the TYPO3 CLI does not offer
			the option to activate/deactivate extensions anymore
		\item Composer-based TYPO3 installations don't need and in fact ignore
			the file "\texttt{ext_emconf.php}" in extensions from now on\newline
			\small(developers may want to keep it for the time being though to
			retain compatibility with TYPO3 instances in non-Composer mode)
		\item Since TYPO3 extensions are Composer packages, they should be
			installed using Composer
		\item See
			\href{https://docs.typo3.org/c/typo3/cms-core/master/en-us/Changelog/11.4/Feature-94996-ConsiderAllComposerInstalledExtensionsAsActive.html}{changelog}
			and
			\href{https://docs.typo3.org/c/typo3/cms-core/master/en-us/Changelog/master/Deprecation-94996-InComposerModeAllExtensionsShouldBeInstalledWithComposer.html}{deprecation}
			for further details
	\end{itemize}
\end{frame}

% ------------------------------------------------------------------------------
