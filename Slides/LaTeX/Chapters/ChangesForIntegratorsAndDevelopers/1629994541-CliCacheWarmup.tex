% ------------------------------------------------------------------------------
% TYPO3 Version 11.4 - What's New (German Version)
%
% @license	Creative Commons BY-NC-SA 3.0
% @link		https://typo3.org/help/documentation/whats-new/
% @language	German
% ------------------------------------------------------------------------------
% Feature | 93436 | Introduce cache:warmup command

\begin{frame}[fragile]
	\frametitle{Sonstiges}
	\framesubtitle{TYPO3 CLI (2)}

	% decrease font size for code listing
	\lstset{basicstyle=\fontsize{9}{11}\ttfamily}

	\begin{itemize}
		\item Ein neuer Befehl wurde der Befehlszeilenschnittstelle (CLI) von TYPO3 hinzugefügt:
\begin{lstlisting}
./bin/typo3 cache:warmup [--group <all|system|di|pages|...>]
\end{lstlisting}
		\item Integratoren können nun alle Caches aufwärmen (Standard) oder
			selektiv durch die verfügbaren Cache-Gruppen
		\item Spezifische Cache-Gruppen können mit der Option
			\texttt{-}\texttt{-group} definiert werden:\newline
			\small\texttt{system}, \texttt{pages}, \texttt{di}, oder \texttt{all} (default: \texttt{all})\normalsize
		\item Hinweis: Es können (noch) nicht \textit{alle} Caches aufgewärmt werden\newline
			\small(z.B. Frontend-Cache, RequireJS-Assets-Cache, Extbase-Caches,
				Fluid-Templates sind derzeit ausgeschlossen)\normalsize
	\end{itemize}
\end{frame}

% ------------------------------------------------------------------------------
